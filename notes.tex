% arara: lualatex
% arara: bibtex
% arara: lualatex
% arara: lualatex
\documentclass[11pt]{article}
\usepackage[dvipsnames]{xcolor}
\usepackage[margin=1.3in]{geometry}
\usepackage{tikz}
\usepackage{csquotes}
\usepackage{eufrak}
\usepackage{proof}
\usepackage[leqno]{amsmath}
\usepackage{amssymb,amsfonts}
\usepackage{mathrsfs}
\usepackage{mathtools}
\usepackage{yfonts}
\usepackage{epigraph}
\usepackage{url}
\usepackage{amsthm}
\usepackage{lipsum}
\usepackage{todonotes}
\usepackage{enumitem}
\usepackage{mathabx}
\newtheorem*{definition}{Definition}
\newtheorem*{notation}{Notation}
\usepackage{hyperref}
\hypersetup{
    colorlinks=true,
    linkcolor=Plum,
    citecolor=RoyalBlue,
    filecolor=black,
    urlcolor=black,
    pdfborderstyle={/S/U/W 0.5}
}
\newcommand{\LSIII}{\hyperref[LS3]{LS3}}
\newcommand{\seq}[1]{\langle #1 \rangle}
\newcommand{\length}[1]{\mathsf{lth}(#1)}

\usepackage{mathpazo}
\usepackage{inconsolata}
\linespread{1.05}
\usepackage[T1,euler-digits]{eulervm}

\newcommand{\nat}{\mathbb{N}}


\author{Ayberk Tosun\\\texttt{tosun2@illinois.edu}}

\title{Notes on Choice Sequences}

\begin{document}
\maketitle
\section{Lawless sequences}

A \emph{lawlike} mathematical objects is one that is determined completely by
a law given in advance. Two lawlike mathematical objects of interest to us are
natural numbers and lawlike sequences of natural numbers, those that can be
completely prescribed via a law.

In contrast to mathematical objects that are lawlike, there are also those that
we we call \emph{lawless} which do not have to be completely prescribed by a
law. Lawless sequences are of special interest to us in that they are the first
manifestation of the rejecting \emph{formalization as a necessary condition} of
mathematicality. They allow us to permit constructions that are not necessarily
\emph{prescribed} into our mathematical world. The prime example of a lawless
sequence is the sequence generated by the iterated flip of a coin
\cite{sep-intuitionism}. No \emph{linguistic} description of this sequence
exists but qualifies as a lawless sequence. Troelstra explains this as follows:
\begin{quote}
  Acceptance of lawless sequences as mathematical objects implies denial of the
  thesis that all mathematical entities should be given to us via a
  linguistic representation. \cite[pg. 11]{troelstra-choice-sequence}
\end{quote}

We may specify the initial segment of a lawless sequence or we can fix values
for a finite set of natural numbers as long as we are not imposing
\emph{general} requirements for the sequence. This is given by the following
axiom:
\begin{equation}
  \forall n.\ \exists \alpha.\ \alpha \in n   \tag{LS1}
\end{equation}

We denote by $\equiv$ intensional identity between lawless sequences i.e.,
$\alpha \equiv \beta$ denotes that $\alpha$ and $\beta$ are are generated by
the same process. Extensional equality between lawless sequences holds if and
only if intensional equality holds. We take its self-evident that intensional
equality is decidable thus rendering extensional equality decidable. This
yields the second axiom:
\begin{equation}
  \forall \alpha.\ \forall \beta.\ \alpha = \beta \lor \alpha \neq \beta
  \tag{LS2}
\end{equation}

\begin{notation}
  We assume that every finite sequence $v_0, \dots v_n$ has a natural
  number \emph{coding} denoted by $\seq{v_0, \dots, v_n}$.
  By $*$ we denote concatenation of codes:
  \[ \seq{v_0, \dots, v_i} * \seq{v_{i+1},  \dots, v_{j}} =
      \seq{v_0, \dots, v_j} \]
\end{notation}

\begin{notation}
  $n \preceq m \coloneqq \exists n.\ n * n' = m$
\end{notation}

\begin{notation}
  $n \prec m \coloneqq n \preceq m \land n \neq m$
\end{notation}

\begin{notation}
  We denote by $\length{s}$, the length function defined as follows:
  \begin{align*}
  \length{\seq{}} &= 0\\
  \length{\seq{x_0, \dots, x_u}} &= u+1.
  \end{align*}
\end{notation}

\begin{notation}
  We denote by $(n)_x$ the inverse of sequence-coding i.e.,
  if $n = \seq{x_0, \dots , x_u}$, then
  \[ (n)_y =
       \begin{cases}
         x_y & \text{if}\ y \le u\\
         0   & \text{otherwise}
       \end{cases} \]
\end{notation}

\begin{notation}
  $n \preceq m \coloneqq \exists n.\ n * n' = m$
\end{notation}

\begin{notation}
  $A\alpha$ denotes the assertion that we have a proof of $A$ for $\alpha$.
\end{notation}

\begin{notation}
  Given a sequence $a$, we denote by $\bar{a}n$ ``first $n$ elements of $a$''
  i.e.,
  \[ \bar{a}u = \begin{cases}
      \seq{} & \text{if}\ u = 0\\
      \seq{a_0, \dots, a_{u \dotdiv 1}} & \text{otherwise}
      \end{cases} \]
  where $\dotdiv$ is the monus operator.
\end{notation}

\begin{notation}
  By $a \in n$ we denote that ``$n$ is a prefix of $a$''
  \[ a \in n \coloneqq \bar{a}(\length{n}) = n \]
\end{notation}

\begin{notation}
  \begin{align*}
  \neq(\alpha, \beta_1, \dots, \beta_n) \qquad &\coloneqq \qquad
    \bigwedge_{i=1}^{n} \alpha \neq B_i\\
  \#(\beta_0, \dots, \beta_n) \qquad &\coloneqq \qquad
    \bigwedge_{i=0}^n \bigwedge_{j=0}^n i \neq j \rightarrow \beta_i \neq \beta_j
  \end{align*}
\end{notation}

Suppose that we are able to assert $A\alpha$. Since $\alpha$ is lawless
sequence, at any stage we are in possesion only of a prefix of $\alpha$. This
means that in asserting $A\alpha$ we did not make use of any information
besides  $\alpha \in n$. So $A$ should hold for any lawless sequence $\beta$
such that $\beta \in n$. We capture this idea through (\LSIII).
\begin{equation}\label{LS3}
  A(\alpha, \bar{\beta}) \land \neq(\alpha, \bar{\beta})
  \rightarrow \exists n.\ \alpha \in n \land
  (\forall \gamma \in n.\ A(\gamma, \bar{\beta}))
  \tag{\LSIII}
\end{equation}
The reason for the requirement of $\neq(\alpha, \beta_1, \dots, \beta_n)$ is
to make precise that sequences whose prefixes we are in possession of are
\emph{distinguishable}. Each process has a distinct identity as a process---as
Troelstra \cite[pg. 11]{troelstra-choice-sequence} calls: its ``individuality''.

\bibliographystyle{alpha}
\bibliography{bibliography.bib}

\end{document}
